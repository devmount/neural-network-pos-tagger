\documentclass[fontsize=12pt, paper=a4, headinclude, twoside=false, parskip=half+, pagesize=auto, numbers=noenddot, open=right, toc=listof, toc=bibliography]{scrreprt}
% PDF-Kompression
\pdfminorversion=5
\pdfobjcompresslevel=1
% Allgemeines
\usepackage[automark]{scrpage2} % Kopf- und Fußzeilen
\usepackage{amsmath,marvosym} % Mathesachen
\usepackage[T1]{fontenc} % Ligaturen, richtige Umlaute im PDF
\usepackage[utf8]{inputenc}% UTF8-Kodierung für Umlaute usw
% Schriften
\usepackage{mathpazo} % Palatino für Mathemodus
\usepackage{setspace} % Zeilenabstand
\onehalfspacing % 1,5 Zeilen
% Schriften-Größen
\setkomafont{chapter}{\Huge\rmfamily} % Überschrift der Ebene
\setkomafont{section}{\Large\rmfamily}
\setkomafont{subsection}{\large\rmfamily}
\setkomafont{subsubsection}{\large\rmfamily}
\setkomafont{chapterentry}{\large\rmfamily} % Überschrift der Ebene in Inhaltsverzeichnis
\setkomafont{descriptionlabel}{\bfseries\rmfamily} % für description Umgebungen
\setkomafont{captionlabel}{\small\bfseries}
\setkomafont{caption}{\small}
% Sprache: Deutsch
\usepackage[english]{babel} % Silbentrennung
\usepackage{csquotes} % quotes
% PDF
\usepackage[english]{hyperref}
% \addto\extrasngerman{% Umbenennung der Kapitel Referenzen
%   \def\subsectionautorefname{Abschnitt}%
%   \def\subsubsectionautorefname{Abschnitt}%
% }
% \usepackage{bookmark}
\usepackage[final]{microtype} % mikrotypographische Optimierungen
\clubpenalty = 10000
\widowpenalty = 10000
\displaywidowpenalty = 10000
\usepackage{url}
\renewcommand*{\UrlFont}{\footnotesize}
\usepackage{pdflscape} % einzelne Seiten drehen können
% Tabellen
\usepackage{multirow} % Tabellen-Zellen über mehrere Zeilen
\usepackage{multicol} % mehre Spalten auf eine Seite
\usepackage{tabularx} % Für Tabellen mit vorgegeben Größen
\usepackage{longtable} % Tabellen über mehrere Seiten
\usepackage{array}
\usepackage{float}
%  Bibliographie
\usepackage{bibgerm} % Umlaute in BibTeX
\usepackage{natbib}
% Bilder
\usepackage{graphicx} % Bilder
\usepackage{color} % Farben
\usepackage{xcolor,colortbl} % Text Hintergrundfarben und Tabellenfarben
\usepackage{varwidth}
\usepackage{changepage}
\graphicspath{{images/}}
\DeclareGraphicsExtensions{.pdf,.png,.jpg} % bevorzuge pdf-Dateien
\usepackage{subcaption}  % mehrere Abbildungen nebeneinander/übereinander
\usepackage[all]{hypcap} % Beim Klicken auf Links zum Bild und nicht zu Caption gehen
% Bildunterschrift
\setcapindent{0em} % kein Einrücken der Caption von Figures und Tabellen
\setcapwidth{0.9\textwidth} % Breite der Caption nur 90% der Textbreite, damit sie sich vom restlichen Text abhebt
\setlength{\abovecaptionskip}{0.2cm} % Abstand der zwischen Bild- und Bildunterschrift
% Quellcode
\usepackage{listings} % für Formatierung in Quelltexten
\usepackage{DejaVuSansMono} % ttfamily font
\usepackage{todonotes}% Todo Notes
\presetkeys{todonotes}{inline}{}
% Bibliography multicoloumn
\usepackage{etoolbox}
\usepackage{relsize}
\patchcmd{\thebibliography}{\list}{\begin{multicols}{2}\small\list}{}{}\appto{\endthebibliography}{\end{multicols}}

\definecolor{gray}{rgb}{0.5,0.5,0.5}
\definecolor{orange}{rgb}{.99,0.5,0}
\definecolor{green}{rgb}{0,0.4,0}
\definecolor{lightgreen}{rgb}{0.7,1,0.7}
\definecolor{codegreen}{rgb}{0,0.6,0}
\definecolor{codegray}{rgb}{0.5,0.5,0.5}
\definecolor{backcolour}{rgb}{0.97,0.97,0.95}

\lstdefinestyle{mystyle}{
  inputencoding={utf8},
  xleftmargin=1em,
  backgroundcolor=\color{backcolour},
  basicstyle=\tiny\ttfamily,
  commentstyle=\color{gray},
  keywordstyle=\color{green}\textbf,
  numberstyle=\tiny\color{codegray},
  stringstyle=\color{orange},
  breakautoindent  = true,
  breakindent      = 2em,
  breaklines       = true,
  postbreak        = ,
  prebreak         = \raisebox{-.8ex}[0ex][0ex]{\Righttorque},
  captionpos=b,
  keepspaces=true,
  numbers=left,
  numbersep=5pt,
  numberstyle=\tiny\ttfamily\color{gray},
  showspaces=false,
  showstringspaces=false,
  showtabs=false,
  tabsize=2,
  literate=%
    {Ö}{{\"O}}1
    {Ä}{{\"A}}1
    {Ü}{{\"U}}1
    {ß}{{\ss}}1
    {ü}{{\"u}}1
    {ä}{{\"a}}1
    {ö}{{\"o}}1
    {~}{{\textasciitilde}}1
}
\lstset{style=mystyle}
% linksbündige Fußboten
\deffootnote{1.5em}{1em}{\makebox[1.5em][l]{\thefootnotemark}}

\typearea{14} % typearea berechnet einen sinnvollen Satzspiegel (das heißt die Seitenränder) siehe auch http://www.ctan.org/pkg/typearea. Diese Berechnung befindet sich am Schluss, damit die Einstellungen oben berücksichtigt werden
\textwidth=440pt % text width

\usepackage{scrhack} % Vermeidung einer Warnung
\usepackage{acronym} % Abkürzungsverzeichnis

% chapter margin
\renewcommand*{\chapterheadstartvskip}{\vspace*{0cm}}
\renewcommand*{\chapterheadendvskip}{\vspace{.5cm}}


% Eigene Befehle %%%%%%%%%%%%%%%%%%%%%%%%%%%%%%%%%%%%%%%%%%%%%%%%%5
% Matrix
\renewcommand*{\i}[1]{%
      {\textit{#1}}%
}
\renewcommand*{\b}[1]{%
      {\textbf{#1}}%
}
\renewcommand*{\tt}[1]{%
      {\footnotesize\texttt{#1}}%
}
\newcommand{\q}[1]{%
      {\enquote{#1}}%
}
\newcommand{\sq}[1]{%
      {\enquote*{#1}}%
}

% break inside a table cell
\newcommand{\br}[3]{%
      {\parbox{#1cm}{#2\\#3\vspace{3pt}}}%
}

\newcommand{\mat}[1]{%
      {\textbf{#1}}%
}
\newcommand{\info}[1]{
      {\colorbox{blue}{ (INFO: #1)}}
}
% Hinweis auf Programme in Datei
\newcommand{\datei}[1]{%
      {\ttfamily{#1}}%
}
\newcommand{\code}[1]{%
      {\footnotesize\ttfamily{\colorbox{gray!20}{\begin{varwidth}{\dimexpr\linewidth-2\fboxsep}#1\end{varwidth}}}}%
}
% bild mit defnierter Breite einfügen
\newcommand{\bild}[4]{
  \begin{figure}[H]
    \centering
      \vspace{1ex}
      \includegraphics[width=#2]{images/#1}
      \caption[#4]{\label{img.#1} #3}
    \vspace{1ex}
  \end{figure}
}
% bild mit defnierter Breite und Leftshift einfügen
\newcommand{\bildl}[5]{
  \begin{figure}[H]
    \centering
      \vspace{1ex}
      \hspace*{#3}
      \includegraphics[width=#2]{images/#1}
      \caption[#5]{\label{img.#1} #4}
    \vspace{1ex}
  \end{figure}
}
% bild mit eigener Breite
\newcommand{\bilda}[3]{
  \begin{figure}[H]
    \centering
      \vspace{1ex}
      \includegraphics{images/#1}
      \caption[#3]{\label{img.#1} #2}
      \vspace{1ex}
  \end{figure}
}
% insert blank page
\newcommand*\BlankPage{\newpage\null\thispagestyle{empty}\newpage}