\documentclass[fontsize=12pt, paper=a4, headinclude, twoside=false, parskip=half+, pagesize=auto, paper=a4, numbers=noenddot, open=right, toc=listof, toc=bibliography]{scrreprt}
% PDF-Kompression
\pdfminorversion=5
\pdfobjcompresslevel=1
% Allgemeines
\usepackage[automark]{scrlayer-scrpage} % Kopf- und Fußzeilen
\usepackage{amsmath,marvosym} % Mathesachen
\usepackage[T1]{fontenc} % Ligaturen, richtige Umlaute im PDF
\usepackage[utf8]{inputenc}% UTF8-Kodierung für Umlaute usw
\hoffset = 35pt
% Schriften
\usepackage{mathpazo} % Palatino für Mathemodus
\usepackage{setspace} % Zeilenabstand
\onehalfspacing % 1,5 Zeilen
% Schriften-Größen
\setkomafont{chapter}{\Huge\rmfamily} % Überschrift der Ebene
\setkomafont{section}{\Large\rmfamily}
\setkomafont{subsection}{\large\rmfamily}
\setkomafont{subsubsection}{\large\rmfamily}
\setkomafont{chapterentry}{\large\rmfamily} % Überschrift der Ebene in Inhaltsverzeichnis
\setkomafont{descriptionlabel}{\bfseries\rmfamily} % für description Umgebungen
\setkomafont{captionlabel}{\small\bfseries}
\setkomafont{caption}{\small}
% language: english
\usepackage[english]{babel} % Silbentrennung
\usepackage{csquotes} % quotes
% PDF
\usepackage[english]{hyperref}
\usepackage[final]{microtype} % mikrotypographische Optimierungen
\clubpenalty = 10000
\widowpenalty = 10000
\displaywidowpenalty = 10000
\usepackage{url}
\usepackage{pdflscape} % einzelne Seiten drehen können
% Tabellen
\usepackage{multirow} % Tabellen-Zellen über mehrere Zeilen
\usepackage{multicol} % mehre Spalten auf eine Seite
\usepackage{tabularx} % Für Tabellen mit vorgegeben Größen
\usepackage{longtable} % Tabellen über mehrere Seiten
\usepackage{tablefootnote} % use footnotes in tables
\newcolumntype{L}[1]{>{\raggedright}p{#1}}
\usepackage{array,booktabs}
\usepackage{float}
%  Bibliographie
\usepackage{bibgerm} % Umlaute in BibTeX
\usepackage{natbib}
% Bilder
\usepackage{graphicx} % Bilder
\usepackage{color} % Farben
\usepackage{xcolor,colortbl} % Text Hintergrundfarben und Tabellenfarben
\usepackage{varwidth}
\usepackage{changepage}
\graphicspath{{images/}}
\DeclareGraphicsExtensions{.pdf,.png,.jpg} % bevorzuge pdf-Dateien
\usepackage{subcaption}  % mehrere Abbildungen nebeneinander/übereinander
\usepackage[all]{hypcap} % Beim Klicken auf Links zum Bild und nicht zu Caption gehen
% Bildunterschrift
\setcapindent{0em} % kein Einrücken der Caption von Figures und Tabellen
\setcapwidth{0.8\textwidth} % Breite der Caption nur 90% der Textbreite, damit sie sich vom restlichen Text abhebt
\setlength{\abovecaptionskip}{0.2cm} % Abstand der zwischen Bild- und Bildunterschrift
% Quellcode
\usepackage{listings} % für Formatierung in Quelltexten
\usepackage{DejaVuSansMono} % ttfamily font
\usepackage{fancyvrb}
% Bibliography multicoloumn
\usepackage{etoolbox}
\usepackage{relsize}
\patchcmd{\thebibliography}{\list}{\begin{multicols}{2}\small\list}{}{}\appto{\endthebibliography}{\end{multicols}}

\definecolor{white}{rgb}{1,1,1}
\definecolor{gray}{rgb}{0.5,0.5,0.5}
\definecolor{orange}{RGB}{237,125,49}
\definecolor{green}{rgb}{0,0.4,0}
\definecolor{lightgreen}{rgb}{0.7,1,0.7}
\definecolor{codegreen}{rgb}{0,0.6,0}
\definecolor{codegray}{rgb}{0.5,0.5,0.5}
\definecolor{backcolour}{rgb}{0.97,0.97,0.95}

\colorlet{punct}{red!60!black}
\definecolor{background}{HTML}{EEEEEE}
\definecolor{delim}{RGB}{20,105,176}
\colorlet{numb}{magenta!60!black}

\lstdefinelanguage{json}{
  backgroundcolor   = \color{white},
  basicstyle        = \fontsize{8}{10}\ttfamily,
  breaklines        = true,
  frame             = none,
  numbers           = none,
  numbersep         = 8pt,
  numberstyle       = \scriptsize,
  showstringspaces  = false,
  stepnumber        = 1,
  xleftmargin       = 0em,
  literate =
   *{0}{{{\color{numb}0}}}{1}
    {1}{{{\color{numb}1}}}{1}
    {2}{{{\color{numb}2}}}{1}
    {3}{{{\color{numb}3}}}{1}
    {4}{{{\color{numb}4}}}{1}
    {5}{{{\color{numb}5}}}{1}
    {6}{{{\color{numb}6}}}{1}
    {7}{{{\color{numb}7}}}{1}
    {8}{{{\color{numb}8}}}{1}
    {9}{{{\color{numb}9}}}{1}
    {:}{{{\color{codegray}{:}}}}{1}
    {,}{{{\color{codegray}{,}}}}{1}
    {\{}{{{\color{codegray}{\{}}}}{1}
    {\}}{{{\color{codegray}{\}}}}}{1}
    {[}{{{\color{codegray}{[}}}}{1}
    {]}{{{\color{codegray}{]}}}}{1},
}
\lstdefinelanguage{plain}{
  backgroundcolor   = \color{white},
  basicstyle        = \fontsize{7}{8}\ttfamily,
  breaklines        = true,
  breakatwhitespace = false,
  frame             = none,
  numbers           = none,
  showstringspaces  = false,
  stepnumber        = 1,
  xleftmargin       = 0em,
}

\lstdefinestyle{master}{
  backgroundcolor   = \color{white},
  basicstyle        = \footnotesize\ttfamily,
  breakautoindent   = true,
  breakindent       = 2em,
  breaklines        = true,
  captionpos        = b,
  commentstyle      = \color{gray},
  inputencoding     = {utf8},
  keepspaces        = true,
  keywordstyle      = \color{green}\textbf,
  numbers           = none,
  numbersep         = 5pt,
  numberstyle       = \footnotesize\ttfamily\color{gray},
  postbreak         = ,
  prebreak          = \raisebox{-.8ex}[0ex][0ex]{\Righttorque},
  showspaces        = false,
  showstringspaces  = false,
  showtabs          = false,
  stringstyle       = \color{orange},
  tabsize           = 2,
  xleftmargin       = 1em,
  literate =
    {á}{{\'a}}1 {é}{{\'e}}1 {í}{{\'i}}1 {ó}{{\'o}}1 {ú}{{\'u}}1
    {Á}{{\'A}}1 {É}{{\'E}}1 {Í}{{\'I}}1 {Ó}{{\'O}}1 {Ú}{{\'U}}1
    {à}{{\`a}}1 {è}{{\`e}}1 {ì}{{\`i}}1 {ò}{{\`o}}1 {ù}{{\`u}}1
    {À}{{\`A}}1 {È}{{\'E}}1 {Ì}{{\`I}}1 {Ò}{{\`O}}1 {Ù}{{\`U}}1
    {ä}{{\"a}}1 {ë}{{\"e}}1 {ï}{{\"i}}1 {ö}{{\"o}}1 {ü}{{\"u}}1
    {Ä}{{\"A}}1 {Ë}{{\"E}}1 {Ï}{{\"I}}1 {Ö}{{\"O}}1 {Ü}{{\"U}}1
    {â}{{\^a}}1 {ê}{{\^e}}1 {î}{{\^i}}1 {ô}{{\^o}}1 {û}{{\^u}}1
    {Â}{{\^A}}1 {Ê}{{\^E}}1 {Î}{{\^I}}1 {Ô}{{\^O}}1 {Û}{{\^U}}1
    {œ}{{\oe}}1 {Œ}{{\OE}}1 {æ}{{\ae}}1 {Æ}{{\AE}}1 {ß}{{\ss}}1
    {ű}{{\H{u}}}1 {Ű}{{\H{U}}}1 {ő}{{\H{o}}}1 {Ő}{{\H{O}}}1
    {ç}{{\c c}}1 {Ç}{{\c C}}1 {ø}{{\o}}1 {å}{{\r a}}1 {Å}{{\r A}}1
    {€}{{\euro}}1 {£}{{\pounds}}1 {«}{{\guillemotleft}}1
    {»}{{\guillemotright}}1 {ñ}{{\~n}}1 {Ñ}{{\~N}}1 {¿}{{?`}}1
    {|}{{\textbar\allowbreak}}1
}
\lstset{style=master}
% linksbündige Fußboten
\deffootnote{1.5em}{1em}{\makebox[1.5em][l]{\thefootnotemark}}

\typearea{14} % typearea berechnet einen sinnvollen Satzspiegel (das heißt die Seitenränder) siehe auch http://www.ctan.org/pkg/typearea. Diese Berechnung befindet sich am Schluss, damit die Einstellungen oben berücksichtigt werden
\textwidth=400pt % text width

\usepackage{scrhack} % Vermeidung einer Warnung
\usepackage{acronym} % Abkürzungsverzeichnis

% chapter margin
\renewcommand*{\chapterheadstartvskip}{\vspace*{0cm}}
\renewcommand*{\chapterheadendvskip}{\vspace{.5cm}}


% Eigene Befehle %%%%%%%%%%%%%%%%%%%%%%%%%%%%%%%%%%%%%%%%%%%%%%%%%5
% Matrix
\renewcommand*{\i}[1]{%
      {\textit{#1}}%
}
\renewcommand*{\b}[1]{%
      {\textbf{#1}}%
}
\renewcommand*{\tt}[1]{%
      {\footnotesize\texttt{#1}}%
}
\newcommand{\q}[1]{%
      {\enquote{#1}}%
}
\newcommand{\sq}[1]{%
      {\enquote*{#1}}%
}

% break inside a table cell
\newcommand{\br}[3]{%
      {\parbox{#1cm}{#2\\#3\vspace{3pt}}}%
}

\newcommand{\mat}[1]{%
      {\textbf{#1}}%
}
\newcommand{\info}[1]{
      {\colorbox{blue}{ (INFO: #1)}}
}
% Hinweis auf Programme in Datei
\newcommand{\datei}[1]{%
      {\ttfamily{#1}}%
}
\newcommand{\code}[1]{%
      {\footnotesize\ttfamily{\colorbox{gray!20}{\begin{varwidth}{\dimexpr\linewidth-2\fboxsep}#1\end{varwidth}}}}%
}
% bild mit defnierter Breite einfügen
\newcommand{\bild}[4]{
  \begin{figure}[H]
    \centering
      \vspace{1ex}
      \includegraphics[width=#2]{images/#1}
      \caption[#4]{\label{img.#1} #3}
    \vspace{1ex}
  \end{figure}
}
% bild mit defnierter Breite und Leftshift einfügen
\newcommand{\bildl}[5]{
  \begin{figure}[H]
    \centering
      \vspace{1ex}
      \hspace*{#3}
      \includegraphics[width=#2]{images/#1}
      \caption[#5]{\label{img.#1} #4}
    \vspace{1ex}
  \end{figure}
}
% bild mit eigener Breite
\newcommand{\bilda}[3]{
  \begin{figure}[H]
    \centering
      \vspace{1ex}
      \includegraphics{images/#1}
      \caption[#3]{\label{img.#1} #2}
      \vspace{1ex}
  \end{figure}
}
% insert blank page
\newcommand*\BlankPage{\newpage\null\thispagestyle{empty}\newpage}
% Name Alex
\newcommand*\Alex{\textsc{Alex}}
% Link formatting
\newcommand{\link}[1]{\scriptsize\url{#1}}
% Footer tt formatting
\newcommand{\ftt}[1]{\tt{\scriptsize{#1}}}
% condensed three dots
\renewcommand*{\dots}{\makebox[.5em][c]{.\hfil.\hfil.}}
% create rules with custom margin and thickness
\newcommand{\trule}{\specialrule{.08em}{0em}{0.2em}}
\newcommand{\drule}{\specialrule{.04em}{0.4em}{0.2em} \specialrule{.04em}{0em}{0em}}
\newcommand{\srule}{\specialrule{.04em}{0em}{0.2em} \specialrule{.04em}{0em}{0em}}
\newcommand{\mrule}{\specialrule{.04em}{0em}{0em}}
\newcommand{\brule}{\specialrule{.08em}{0em}{1em}}
